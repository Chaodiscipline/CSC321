%%%%%%%%%%%%%%%%%%%%%%%%%%%%%%%%%%%%%%%%%
% Programming/Coding Assignment
% LaTeX Template
%
% This template has been downloaded from:
% http://www.latextemplates.com
%
% Original author:
% Ted Pavlic (http://www.tedpavlic.com)
%
% Note:
% The \lipsum[#] commands throughout this template generate dummy text
% to fill the template out. These commands should all be removed when 
% writing assignment content.
%
% This template uses a Perl script as an example snippet of code, most other
% languages are also usable. Configure them in the "CODE INCLUSION 
% CONFIGURATION" section.
%
%%%%%%%%%%%%%%%%%%%%%%%%%%%%%%%%%%%%%%%%%

%----------------------------------------------------------------------------------------
%	PACKAGES AND OTHER DOCUMENT CONFIGURATIONS
%----------------------------------------------------------------------------------------

\documentclass{article}

\usepackage{fancyhdr} % Required for custom headers
\usepackage{lastpage} % Required to determine the last page for the footer
\usepackage{extramarks} % Required for headers and footers
\usepackage[usenames,dvipsnames]{color} % Required for custom colors
\usepackage{graphicx} % Required to insert images
\usepackage{listings} % Required for insertion of code
\usepackage{courier} % Required for the courier font
\usepackage{lipsum} % Used for inserting dummy 'Lorem ipsum' text into the template
\usepackage{pgfplotstable} % Import csv data
\usepackage{amsmath} % Math equation tools
\usepackage{hyperref} % Create references\pgfplotsset{compat=1.12}
\usepackage{pgf} % Loops in latex
\usepackage{bookmark}
\usepackage{booktabs}
\usepackage{float}

% Margins
\topmargin=-0.45in
\evensidemargin=0in
\oddsidemargin=0in
\textwidth=6.5in
\textheight=9.0in
\headsep=0.25in

\linespread{1.1} % Line spacing

% Set up the header and footer
\pagestyle{fancy}
\lhead{\hmwkAuthorName} % Top left header
\chead{\hmwkClass\ (\hmwkClassTime): \hmwkTitle} % Top center head
%\rhead{\firstxmark} % Top right header
\lfoot{\lastxmark} % Bottom left footer
\cfoot{} % Bottom center footer
\rfoot{Page\ \thepage\ of\ \protect\pageref{LastPage}} % Bottom right footer
\renewcommand\headrulewidth{0.4pt} % Size of the header rule
\renewcommand\footrulewidth{0.4pt} % Size of the footer rule

\DeclareGraphicsExtensions{.eps,.pdf,.png,.jpg}

\setlength\parindent{0pt} % Removes all indentation from paragraphs

%----------------------------------------------------------------------------------------
%	CODE INCLUSION CONFIGURATION
%----------------------------------------------------------------------------------------

\definecolor{MyDarkGreen}{rgb}{0.0,0.4,0.0} % This is the color used for comments
\lstloadlanguages{Perl} % Load Perl syntax for listings, for a list of other languages supported see: ftp://ftp.tex.ac.uk/tex-archive/macros/latex/contrib/listings/listings.pdf
\lstset{language=Perl, % Use Perl in this example
        frame=single, % Single frame around code
        basicstyle=\small\ttfamily, % Use small true type font
        keywordstyle=[1]\color{Blue}\bf, % Perl functions bold and blue
        keywordstyle=[2]\color{Purple}, % Perl function arguments purple
        keywordstyle=[3]\color{Blue}\underbar, % Custom functions underlined and blue
        identifierstyle=, % Nothing special about identifiers                                         
        commentstyle=\usefont{T1}{pcr}{m}{sl}\color{MyDarkGreen}\small, % Comments small dark green courier font
        stringstyle=\color{Purple}, % Strings are purple
        showstringspaces=false, % Don't put marks in string spaces
        tabsize=5, % 5 spaces per tab
        %
        % Put standard Perl functions not included in the default language here
        morekeywords={rand},
        %
        % Put Perl function parameters here
        morekeywords=[2]{on, off, interp},
        %
        % Put user defined functions here
        morekeywords=[3]{test},
       	%
        morecomment=[l][\color{Blue}]{...}, % Line continuation (...) like blue comment
        numbers=left, % Line numbers on left
        firstnumber=1, % Line numbers start with line 1
        numberstyle=\tiny\color{Blue}, % Line numbers are blue and small
        stepnumber=5 % Line numbers go in steps of 5
}

% Creates a new command to include a perl script, the first parameter is the filename of the script (without .pl), the second parameter is the caption
\newcommand{\perlscript}[2]{
\begin{itemize}
\item[]\lstinputlisting[caption=#2,label=#1]{#1.pl}
\end{itemize}
}

%----------------------------------------------------------------------------------------
%	DOCUMENT STRUCTURE COMMANDS
%	Skip this unless you know what you're doing
%----------------------------------------------------------------------------------------

% Header and footer for when a page split occurs within a problem environment
\newcommand{\enterProblemHeader}[1]{
%\nobreak\extramarks{#1}{#1 continued on next page\ldots}\nobreak
%\nobreak\extramarks{#1 (continued)}{#1 continued on next page\ldots}\nobreak
}

% Header and footer for when a page split occurs between problem environments
\newcommand{\exitProblemHeader}[1]{
%\nobreak\extramarks{#1 (continued)}{#1 continued on next page\ldots}\nobreak
%\nobreak\extramarks{#1}{}\nobreak
}

\setcounter{secnumdepth}{0} % Removes default section numbers
\newcounter{homeworkProblemCounter} % Creates a counter to keep track of the number of problems
\setcounter{homeworkProblemCounter}{0}

\newcommand{\homeworkProblemName}{}
\newenvironment{homeworkProblem}[1][Part \arabic{homeworkProblemCounter}]{ % Makes a new environment called homeworkProblem which takes 1 argument (custom name) but the default is "Problem #"
\def\sectionautorefname{Part}
\refstepcounter{homeworkProblemCounter}%
\renewcommand{\homeworkProblemName}{#1} % Assign \homeworkProblemName the name of the problem
\section{\homeworkProblemName} % Make a section in the document with the custom problem count
\enterProblemHeader{\homeworkProblemName} % Header and footer within the environment
}{
\exitProblemHeader{\homeworkProblemName} % Header and footer after the environment
}

\newcommand{\problemAnswer}[1]{ % Defines the problem answer command with the content as the only argument
\noindent\framebox[\columnwidth][c]{\begin{minipage}{0.98\columnwidth}#1\end{minipage}} % Makes the box around the problem answer and puts the content inside
}

\newcommand{\homeworkSectionName}{}
\newenvironment{homeworkSection}[1]{ % New environment for sections within homework problems, takes 1 argument - the name of the section
\renewcommand{\homeworkSectionName}{#1} % Assign \homeworkSectionName to the name of the section from the environment argument
\subsection{\homeworkSectionName} % Make a subsection with the custom name of the subsection
\enterProblemHeader{\homeworkProblemName\ [\homeworkSectionName]} % Header and footer within the environment
}{
\enterProblemHeader{\homeworkProblemName} % Header and footer after the environment
}

%----------------------------------------------------------------------------------------
%	NAME AND CLASS SECTION
%----------------------------------------------------------------------------------------

\newcommand{\hmwkTitle}{Assignment 1} % Assignment title
\newcommand{\hmwkDueDate}{Wednesday,\ February\ 3,\ 2016} % Due date
\newcommand{\hmwkClass}{CSC321} % Course/class
\newcommand{\hmwkClassTime}{L0101} % Class/lecture time
\newcommand{\hmwkAuthorName}{Davi Frossard} % Your name

%----------------------------------------------------------------------------------------
%	TITLE PAGE
%----------------------------------------------------------------------------------------

\title{
\vspace{2in}
\textmd{\textbf{\hmwkClass:\ \hmwkTitle}}\\
\normalsize\vspace{0.1in}\small{Due\ on\ \hmwkDueDate}\\
\vspace{0.1in}
\vspace{3in}
}

\author{\textbf{\hmwkAuthorName}}
%\date{} % Insert date here if you want it to appear below your name

%----------------------------------------------------------------------------------------

\begin{document}

\maketitle
\clearpage


%----------------------------------------------------------------------------------------
%	PART 1
%----------------------------------------------------------------------------------------

\begin{homeworkProblem}

The dataset consists of 3167 links to pictures of 15 actors and 3169 links to picture of 15 actresses, each photo and face is uniquely identified and contains the coordinates for the face on the picture. The dimensions of the pictures vary considerably and some links are broken or have corrupted images.

\begin{figure}[H]
    \centering
    \includegraphics[width=0.5\columnwidth]{results/part_1/photos/1}
    \caption{Photo of Gerard Butler}
\end{figure}

\begin{figure}[H]
    \centering
    \includegraphics[width=0.5\columnwidth]{results/part_1/photos/2}
    \caption{Photo of Leonardo DiCaprio}
\end{figure}


\begin{figure}[H]
    \centering
    \includegraphics[width=0.5\columnwidth]{results/part_1/photos/3}
    \caption{Photo of Jason Statham}
\end{figure}

If we sample 15 faces of the same actor, we get the statistics for the dimensions shown in \autoref{t1:stats}, which shows great dissimilarity in terms of dimensions. If we then resize them to the same dimensions and overlay them, we get the picture shown in \autoref{t1:overlay}. From it we can see that the features are not perfectly aligned and there are some considerable outliers, however we can still identify the artist.

\begin{figure}[H]
    \centering
    \includegraphics[width=0.5\columnwidth]{results/part_1/face_overlay/face_overlay}
    \caption{Overlay of 15 faces of Richard Madden}
    \label{t1:overlay}
\end{figure}

\begin{table}[H]
\begin{center}
\pgfplotstabletypeset[
    header=false,
    every head row/.style={
        before row=\toprule
    },
    every last row/.style={
        after row=\bottomrule
    },
    display columns/0/.style={column name={}},
    display columns/1/.style={column name={$Width$}},
    display columns/2/.style={column name={$Length$}},
    /pgf/number format/precision=4,
    create on use/newcol/.style={
        create col/set list={$\sigma$,$\bar{\chi}$}
    },
    columns/newcol/.style={string type},
    columns={newcol,0,1}
]{results/part_1/faces_statistics.csv}
\end{center}
\caption{Standard deviation and average of faces dimensions}
\label{t1:stats} 
\end{table}

\clearpage
\end{homeworkProblem}
%----------------------------------------------------------------------------------------


%----------------------------------------------------------------------------------------
%	PART 2
%----------------------------------------------------------------------------------------

\begin{homeworkProblem}
\label{t2}

To produce the train, validation and test sets we first invoke a function to download the needed amount of pictures for each actor, it downloads the images, crops the faces out, resizes them and converts to grayscale. It then saves the resized faces to disk (for reproducibility in the future) and also returns them as numpy arrays. Another function is then called, this one takes the array of faces and slices them according to the parameters used, by default 100 for training, 10 for validation, 10 for test. Finally it returns two numpy arrays for each set, one containing the image data and the other one the class of each image (the actor portrayed).

\clearpage
\end{homeworkProblem}
%----------------------------------------------------------------------------------------


%----------------------------------------------------------------------------------------
%	PART 3
%----------------------------------------------------------------------------------------

\begin{homeworkProblem}
\label{t3}

Given a face we want to determine the actor portrayed, to do that we use \textit{K}-Nearest Neighbors. First we have to determine the optimum value of nearest neighbors to consider (\textit{K}). We first sweep values from 1 to the size of the datasets, which returns \autoref{t3:k_sens}. We then select the values of \textit{K} that gives the better results (\autoref{t3:k_eval}) and test it against the test set, which gives us \autoref{t3:k_test}.

\begin{figure}[H]
    \centering
    \includegraphics[width=\columnwidth]{results/part_3/k_sensitivity/k_sensitivity}
    \caption{\textit{K} sensitivity test on validation set}
    \label{t3:k_sens}
\end{figure}

\begin{table}[H]
\begin{center}
\pgfplotstabletypeset[
    col sep=space,
    string type,
    display columns/0/.style={column name=\textbf{\textit{K}}, column type={c}},
    display columns/1/.style={column name=\textbf{Validation Errors (\%)}, column type={c}},
    every head row/.style={before row=\toprule},
    every last row/.style={after row=\bottomrule}
    ]{results/part_3/eval_performance.csv}
\end{center}
\caption{Best performance on validation set}
\label{t3:k_eval} 
\end{table}

\begin{table}[H]
\begin{center}
\pgfplotstabletypeset[
    col sep=space,
    string type,
    display columns/0/.style={column name=\textbf{\textit{K}}, column type={c}},
    display columns/1/.style={column name=\textbf{Test Errors (\%)}, column type={c}},
    every head row/.style={before row=\toprule},
    every last row/.style={after row=\bottomrule}
    ]{results/part_3/test_performance.csv}
\end{center}
\caption{Evaluation of \textit{K} on test set.}
\label{t3:k_test} 
\end{table}

Some failure cases on the test set are shown below, for each sample its five nearest neighbours are shown in order on the right, correct candidates are in greed, incorrect in red: 
\foreach \x in {1,3,5,7,9}{
\begin{figure}[H]
    \centering
    \includegraphics[width=0.5\columnwidth]{results/part_3/mislabels/\x}
\end{figure}
}


\clearpage
\end{homeworkProblem}
%----------------------------------------------------------------------------------------



%----------------------------------------------------------------------------------------
%	PART 4
%----------------------------------------------------------------------------------------

\begin{homeworkProblem}
\label{t4}

If we sweep the values of \textit{K} from 1 to the length of training set with a step of 1 until 10 and a step of 10 afterwards, we get the graph depicted in \autoref{t4:k_sweep}.

\begin{figure}[H]
    \centering
    \includegraphics[width=\columnwidth]{results/part_4/k_sweep/k_sweep}
    \caption{\textit{K} sensitivity test across all sets}
    \label{t4:k_sweep}
\end{figure}

For lower values of \textit{K} (specially 1) the model is over-fitted to the training set, since the nearest neighbor of a sample will always be itself. As \textit{K} increases the decision boundary becomes more and more smoothed, which leads to under-fitting. Its worth noting how when \textit{K} equals the size of the training set the performance replicates that of $\textit{K} = 1$, this happens because the way the sets are made (discussed in \autoref{t2}) ensures that there are no dominant class on the training set, that is, all classes have the same amount of samples. This way, when \textit{K} approaches the size of the training set all classes will have the same amount of votes and therefore it will just be assigned to the one of its nearest neighbor.
\clearpage
\end{homeworkProblem}
%----------------------------------------------------------------------------------------



%----------------------------------------------------------------------------------------
%	PART 5
%----------------------------------------------------------------------------------------

\begin{homeworkProblem}
\label{t5}

Now, given a face we want to determine the gender of the actor portrayed. Same as in \autoref{t3}, we determine the optimum value of \textit{K} by sweeping the values on the validation set, which returns \autoref{t5:k_sens}. We then compare the optimum values against the test set, which gives us \autoref{t5:k_test}.

\begin{figure}[H]
    \centering
    \includegraphics[width=\columnwidth]{results/part_5/k_sensitivity/k_sensitivity}
    \caption{\textit{K} sensitivity test on validation set.}
    \label{t5:k_sens}
\end{figure}

\begin{table}[H]
\begin{center}
\pgfplotstabletypeset[
    col sep=space,
    string type,
    display columns/0/.style={column name=\textbf{\textit{K}}, column type={c}},
    display columns/1/.style={column name=\textbf{Validation Errors (\%)}, column type={c}},
    every head row/.style={before row=\toprule},
    every last row/.style={after row=\bottomrule}
    ]{results/part_5/eval_performance.csv}
\end{center}
\caption{Best performance on validation set.}
\label{t5:k_eval} 
\end{table}

\begin{table}[H]
\begin{center}
\pgfplotstabletypeset[
    col sep=space,
    string type,
    display columns/0/.style={column name=\textbf{\textit{K}}, column type={c}},
    display columns/1/.style={column name=\textbf{Test Errors (\%)}, column type={c}},
    every head row/.style={before row=\toprule},
    every last row/.style={after row=\bottomrule}
    ]{results/part_5/test_performance.csv}
\end{center}
\caption{Best performance on test set.}
\label{t5:k_test} 
\end{table}

Some failures are shown below, following the same model of \autoref{t3}:
\foreach \x in {1,3,4,5,6}{
\begin{figure}[H]
    \centering
    \includegraphics[width=0.5\columnwidth]{results/part_5/mislabels/\x}
\end{figure}
}


\clearpage
\end{homeworkProblem}
%----------------------------------------------------------------------------------------



%----------------------------------------------------------------------------------------
%	PART 6
%----------------------------------------------------------------------------------------

\begin{homeworkProblem}

We now repeat \autoref{t5}, however for the validation and test set we use faces of actors that are not present in the training set, following the same process we get:

\begin{figure}[H]
    \centering
    \includegraphics[width=\columnwidth]{results/part_6/k_sensitivity/k_sensitivity}
    \caption{\textit{K} sensitivity test on validation set.}
    \label{t6:k_sens}
\end{figure}

\begin{table}[H]
\begin{center}
\pgfplotstabletypeset[
    col sep=space,
    string type,
    display columns/0/.style={column name=\textbf{\textit{K}}, column type={c}},
    display columns/1/.style={column name=\textbf{Validation Errors (\%)}, column type={c}},
    every head row/.style={before row=\toprule},
    every last row/.style={after row=\bottomrule}
    ]{results/part_6/eval_performance.csv}
\end{center}
\caption{Best performance on validation set.}
\label{t6:k_eval} 
\end{table}

\begin{table}[H]
\begin{center}
\pgfplotstabletypeset[
    col sep=space,
    string type,
    display columns/0/.style={column name=\textbf{\textit{K}}, column type={c}},
    display columns/1/.style={column name=\textbf{Test Errors (\%)}, column type={c}},
    every head row/.style={before row=\toprule},
    every last row/.style={after row=\bottomrule}
    ]{results/part_6/test_performance.csv}
\end{center}
\caption{Best performance on test set.}
\label{t6:k_test} 
\end{table}

Some failures are shown below, following the same model of \autoref{t3}:
\foreach \x in {1,4,7,10,13}{
\begin{figure}[H]
    \centering
    \includegraphics[width=0.5\columnwidth]{results/part_6/mislabels/\x}
\end{figure}
}

To better illustrate how this task differs from \autoref{t5} and \autoref{t3}, we do a sweep of \textit{K} across all sets, like we did in \autoref{t4}, we get the result shown in \autoref{t6:k_sweep}.

\begin{figure}[H]
    \centering
    \includegraphics[width=\columnwidth]{results/part_6/k_sweep/k_sweep}
    \caption{\textit{K} sensitivity test across all sets}
    \label{t6:k_sweep}
\end{figure}

Its noticeable how the accuracy of the training set differ far more the test and validation sets than in \autoref{t4:k_sweep}. This happens because we were classifying actors that were in all sets, this way each sample outside the training set had a high likelihood of having a sample in the training set that shared many features, since it is the same person. However, now the actors on the training set are not the same as the ones on the validation and test sets, so we start to rely on inherent features of men and women to make the classification, which poses as a harder task, evidenced by how the performance changed from \autoref{t5:k_test} to \autoref{t6:k_test}.

\clearpage
\end{homeworkProblem}
%----------------------------------------------------------------------------------------

\end{document}
